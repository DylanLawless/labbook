%%%%%%%%%%%%%%%%%%
%		Labbook Settings
%		Dylan Lawless
%		2020
%%%%%%%%%%%%%%%%%%

\documentclass[a4paper,
								12pt,
								fleqn]{book}

\usepackage{listings} % for writing code verbatim. use \begin{lstlisting}
\usepackage{ccaption} % http://ctan.org/pkg/ccaption. This allows the legend placement on next page. May have problem with hyperref.
\usepackage[T1]{fontenc}
\usepackage[utf8]{inputenc}
\usepackage[english]{babel}
%\usepackage[french,german,english]{babel}

%%%%%%%%%%%%%%%%%%
%%%%	Index	%%%%
%%%%%%%%%%%%%%%%%%
\usepackage{imakeidx} % this one allows texmaker to make the index file. I believe the similar package \makeidx requires a separate step to run the index command. 
\makeindex %  this enables indexing commands. \printindex is used at the end of the main file after the bib. \printindex Can be in the contexts file instead too. 

% This is used to show the indext neatly on the page where it was sourced from. From someone on StackOverflow. It is used instead of the % \usepackage{showidx} which turns off the normal index printed at the end. 
\let\oldindex\index
\renewcommand{\index}[1]{\oldindex{#1}\marginpar{\small#1}}

%%%%%%%%%%%%%%%%%%
%%%%	Spacing	%%%%
%%%%%%%%%%%%%%%%%%
\usepackage{setspace} % increase interline spacing slightly
\setstretch{1.1} % my preferable size.

\makeatletter
\setlength{\@fptop}{0pt}  % for aligning all floating figures/tables etc... to the top margin
\makeatother

%%%%%%%%%%%%%%%%%%
%%%%	Table of contents	%%%%
%%%%%%%%%%%%%%%%%%
\setcounter{tocdepth}{4} %Sets the depth for TOC
\setcounter{secnumdepth}{0} % \setcounter{secnumdepth}{4} %Sets the numbering depth. -1 part, 0 chapter, 1 section, 2 subsection,  3 subsubsection, 4 paragraph, 5 subparagraph.

%The following three lines are included to add "Chapter" in front of the chapter number in the table of contents. Otherwise it would just read "1... 2.. etc"
\usepackage{tocloft,calc}
%\renewcommand{\cftchappresnum}{Chapter }
%\AtBeginDocument{\addtolength\cftchapnumwidth{\widthof{\bfseries Chapter }}}

%%%%%%%%%%%%%%%%%%
%%%% Formatting	%%%%
%%%%%%%%%%%%%%%%%%
\usepackage{amsmath} % For typesetting math
\usepackage{siunitx} %scientific units for measurements
\DeclareSIUnit\Molar{\textsc{m}} %scientific units for measurements molarity

%%%%%%%%%%%%%%%%%%
%%%%	Biblio	%%%%
%%%%%%%%%%%%%%%%%%
\usepackage[numbers,sectionbib,sort&compress]{natbib}  % sectionbib if to add bib per-chapter in conjuction with \usepackage{chapterbib}. Numbers is changing from the default author year to numbers. (Items in bibliography sorted in order cited). 
% natbib used in combination with style plainnat in the biblio file.

\setlength{\bibsep}{0.5pt}
\usepackage{chapterbib} % to add per-chapter bibliography.
\usepackage{notoccite}  % This is allowing citations to list in number of appearance 

%\usepackage[nottoc,numbib]{tocbibind}
	%Au­to­mat­i­cally adds the bib­li­og­ra­phy and/or the in­dex and/or the con­tents, etc., to the Ta­ble of Con­tents list­ing.
	
%%%%%%%%%%%%%%%%%%
%%%% Tables	%%%%
%%%%%%%%%%%%%%%%%%
\usepackage{tabularx}  % The same arguments as tabular*, but modifies the widths of certain columns, rather than the inter column space, to set a table with the requested total width. The columns that may stretch are marked with the new token X in the preamble argument.
\newcolumntype{Z}{ >{\centering\arraybackslash}X } % This preamble is used for centering in tabularx and X columns, i.e. the cntered heading box on a multi-column table. 

\usepackage{multirow} % This package is used in tables to make a column that spans vertically over several rows. Example use: for a coulm, 4 rows tall, with the work Test; {\multirow{4}{*}{Test}}.

\usepackage{ltablex} % this is to split table over multiple pages.

%%%%%%%%%%%%%%%%%%
%%%%	Drawings	%%%%
%%%%%%%%%%%%%%%%%%
\usepackage{qtree} 
	% this is for decision trees, or branching illustrastion etc. An example would be: $ \Tree [.S [ I [.VP use [.NP {\LaTeX} ]]]] 

\usepackage{tikz} %overlay caption and image or scale bar. 

\usepackage{textcomp}
	% to draw arrows in text mode use either command 
		% A$\,\to\,$B
		% A\textrightarrow B
		
\usepackage{wrapfig}

\usepackage{xymtexpdf} % chemical structures: for a simple ring structure use \cyclohexanev{}. This package's use of ; clashes with French for some reason. Apparently using a hack will work around: \begingroup \shorthandoff{;}  "formula here " \endgroup
		
%%%%%%%%%%%%%%%%%%
%%%%	Other formatting and color use	%%%%
%%%%%%%%%%%%%%%%%%
\usepackage{graphicx,xcolor}  % Required for including images
\graphicspath{{images/}}
\usepackage{courier} % to print font use \texttt{} 
\usepackage{subfig}
\usepackage{booktabs}
\usepackage{lipsum}
\usepackage{microtype}
\usepackage{url}
\usepackage[final]{pdfpages}

\usepackage{listings}
\lstset{language=[LaTeX]Tex,tabsize=4, basicstyle=\scriptsize\ttfamily, showstringspaces=false, numbers=left, numberstyle=\tiny, numbersep=10pt, breaklines=true, breakautoindent=true, breakindent=10pt}

\usepackage{hyperref}
\newcounter{dummy} % before a label \refstepcounter{dummy} is used to reset the counter and stop ref from linking to top of chapter.
\hypersetup{pdfborder={0 0 0},
	colorlinks=true,
	linkcolor={blue!50!black},
	citecolor={blue!50!black},
	urlcolor=black}
\urlstyle{same}

\makeatletter
\def\cleardoublepage{\clearpage\if@twoside \ifodd\c@page\else
    \hbox{}
    \thispagestyle{empty}
    \newpage
    \if@twocolumn\hbox{}\newpage\fi\fi\fi}
\makeatother \clearpage{\pagestyle{plain}\cleardoublepage}

%%%%%%%%%%%%%%%%%%
%%%%% Chapter Header %%%%
%%%%%%%%%%%%%%%%%%
\usepackage{color}
\usepackage{tikz}
\usepackage[compact]{titlesec} %\usepackage[explicit]{titlesec}
 
%\newcommand*\chapterlabel{}
%%\renewcommand{\thechapter}{\Roman{chapter}}
%\titleformat{\chapter}[display]  % type (section,chapter,etc...) to vary,  shape (eg display-type)
%	{\normalfont\bfseries\Huge} % format of the chapter
%	{\gdef\chapterlabel{\thechapter\ }}     % the label 
% 	{0pt} % separation between label and chapter-title
% 	  {\begin{tikzpicture}[remember picture,overlay]
%    \node[yshift=-8cm] at (current page.north west)
%      {\begin{tikzpicture}[remember picture, overlay]
%        \draw[fill=black] (0,0) rectangle(35.5mm,15mm);
%        \node[anchor=north east,yshift=-7.2cm,xshift=34mm,minimum height=30mm,inner sep=0mm] at (current %page.north west)
%        {\parbox[top][30mm][t]{15mm}{\raggedleft $\phantom{\textrm{l}}$\color{white}\chapterlabel}};  %the black l is just %to get better base-line alingement
%        \node[anchor=north west,yshift=-7.2cm,xshift=37mm,text width=\textwidth,minimum height=30mm,inner %sep=0mm] at (current page.north west)
%              {\parbox[top][30mm][t]{\textwidth}{\color{black}#1}};
%       \end{tikzpicture}
%      };
%   \end{tikzpicture}
%   \gdef\chapterlabel{}
%  } % code before the title body

% \titleformat{\chapter}[display]
%   {\normalfont\bfseries}{\chaptertitlename\ \thechapter}{20pt}{}
  
\titleformat{\section}[display]
  {\normalfont\bfseries}{\sectiontitlename\ \thesection}{20pt}{}

\titleformat{\subsection}[display]
  {\normalfont\bfseries}{\sectiontitlename\ \thesection}{20pt}{}
    
\titlespacing*{\chapter}{0pt}{0pt}{0pt} % I want to remove excess spacing 
%\titlespacing*{\section}{0pt}{13.2pt}{*0}  % 13.2pt is line spacing for a text with 11pt font size
%\titlespacing*{\subsection}{0pt}{13.2pt}{*0}
%\titlespacing*{\subsubsection}{0pt}{13.2pt}{*0}

%\newcounter{myparts}
%\newcommand*\partlabel{}
%\titleformat{\part}[display]  % type (section,chapter,etc...) to vary,  shape (eg display-type)
%	{\normalfont\bfseries\Huge} % format of the part
%	{\gdef\partlabel{\thepart\ }}     % the label 
% 	{0pt} % separation between label and part-title
% 	  {\setlength{\unitlength}{20mm}
%	  \addtocounter{myparts}{1}
%	  \begin{tikzpicture}[remember picture,overlay]
%    \node[anchor=north west,xshift=-65mm,yshift=-6.9cm-\value{myparts}*20mm] at (current page.north east) % for %unknown reasons: 3mm missing -> 65 instead of 62
%      {\begin{tikzpicture}[remember picture, overlay]
%        \draw[fill=black] (0,0) rectangle(62mm,20mm);   % -\value{myparts}\unitlength
%        \node[anchor=north west,yshift=-6.1cm-\value{myparts}*20mm,xshift=-60.5mm,minimum height=30mm,inner %sep=0mm] at (current page.north east)
%        {\parbox[top][30mm][t]{55mm}{\raggedright \color{white}Part \partlabel $\phantom{\textrm{l}}$}};  %the phantom l %is just to get better base-line alingement
%        \node[anchor=north east,yshift=-6.1cm-\value{myparts}*20mm,xshift=-63.5mm,text width=\textwidth,minimum height=30mm,inner sep=0mm] at (current page.north east)
%              {\parbox[top][30mm][t]{\textwidth}{\raggedleft \color{black}#1}};
%       \end{tikzpicture}
%      };
%   \end{tikzpicture}
%   \gdef\partlabel{}
%  } % code before the title body

%%%%%%%%%%%%%%%%%%
%%%%% New settings for labbook %%%%
%%%%%%%%%%%%%%%%%%
\usepackage[bottom=10em, margin=4cm]{geometry} % Marging make the horizontal space smaller for quick reading. Bottom size reduces the whitespace at the bottom of the page so more text can fit

\usepackage{marginnote} % used after loading geometry. 

%\usepackage[osf]{mathpazo} % Palatino as the main font
%\linespread{1.05}\selectfont % Palatino needs some extra spacing, here 5% extra
%\usepackage[scaled=.88]{beramono} % Bera-Monospace
%\usepackage[scaled=.86]{berasans} % Bera Sans-Serif

%\usepackage{booktabs,array} % Packages for tables

\usepackage{etoolbox}
%\usepackage[norule]{footmisc} % Removes the horizontal rule from footnotes
\usepackage{lastpage} % Counts the number of pages of the document

% Print chapters on same page instead of starting on fresh pages. Use \usepackage{etoolbox} and remove the clear double page after chapters
%\makeatletter
%\patchcmd{\chapter}{\if@openright\cleardoublepage\else\clearpage\fi}{}{}{}
%\makeatother

% Paragraph indent setting
% \setlength{\parindent}{0pt}

% I want a page header with a hyperlink to jump to the most current entry. fancyhdr will allow this kind of header. 
\usepackage{fancyhdr} 
\fancyhf{}
\cfoot{\thepage}
\pagestyle{fancy} 

%%%%%%%%%%%%%%%%%%
%%%%	Make a command for labelling the daily project title	%%%%
%%%%%%%%%%%%%%%%%%
% My projects are called "A" and "B"

\newcommand{\projectA}[1]{{\color{purple} \bf{projectA}}}
\newcommand{\projectB}[1]{{\color{teal}{projectB}}}

\newcommand{\labday}{\section}% \labday will be replaced with \section. However, the "labday" does not show up as a heading in the texmaker Structure panel. 





%%%% Other	%%%%
%\usepackage[dvipsnames]{xcolor}  % Allows the definition of hex colors
%\definecolor{titleblue}{rgb}{0.16,0.24,0.64} % Custom color for the title on the title page
%\definecolor{linkcolor}{rgb}{0,0,0.42} % Custom color for links - dark blue at the moment

%\usepackage{scrlayer-scrpage} % part of the KOMA-script bundle. Using this package to customize the headers and footers is recommended with every KOMA-script class. The package can be used with the standard classes as well.

%\addtokomafont{title}{\Huge\color{titleblue}} % Titles in custom blue color
%\addtokomafont{chapter}{\color{OliveGreen}} % Lab dates in olive green
%\addtokomafont{section}{\color{Sepia}} % Sections in sepia
%\addtokomafont{pagehead}{\normalfont\sffamily\color{gray}} % Header text in gray and sans serif
%\addtokomafont{caption}{\footnotesize\itshape} % Small italic font size for captions
%\addtokomafont{captionlabel}{\upshape\bfseries} % Bold for caption labels
%\addtokomafont{descriptionlabel}{\rmfamily}
%\setcapwidth[r]{10cm} % Right align caption text
%\setkomafont{footnote}{\sffamily} % Footnotes in sans serif

% \deffootnote[4cm]{4cm}{1em}{\textsuperscript{\thefootnotemark}} % Indent footnotes to line up with text

% \DeclareFixedFont{\textcap}{T1}{phv}{bx}{n}{1.5cm} % Font for main title: Helvetica 1.5 cm
% \DeclareFixedFont{\textaut}{T1}{phv}{bx}{n}{0.8cm} % Font for author name: Helvetica 0.8 cm

%\usepackage[nouppercase,headsepline]{scrpage2} % Provides headers and footers configuration
%\pagestyle{scrheadings} % Print the headers and footers on all pages
%\clearscrheadfoot % Clean old definitions if they exist

%\automark[chapter]{chapter}
%\ohead{\headmark} % Prints outer header

%\setlength{\headheight}{25pt} % Makes the header take up a bit of extra space for aesthetics
%\setheadsepline{.4pt} % Creates a thin rule under the header
%\addtokomafont{headsepline}{\color{lightgray}} % Colors the rule under the header light gray

%\ofoot[\normalfont\normalcolor{\thepage\ |\  \pageref{LastPage}}]{\normalfont\normalcolor{\thepage\ |\  %\pageref{LastPage}}} % Creates an outer footer of: "current page | total pages"

% These lines make it so each new lab day directly follows the previous one i.e. does not start on a new page - comment them out to separate lab days on new pages
%\makeatletter
%\patchcmd{\addchap}{\if@openright\cleardoublepage\else\clearpage\fi}{\par}{}{}
%\makeatother
%\renewcommand*{\chapterpagestyle}{scrheadings}

% These lines make it so every figure and equation in the document is numbered consecutively rather than restarting at 1 for each lab day - comment them out to remove this behavior
%\usepackage{chngcntr}
%\counterwithout{figure}{labday}
%\counterwithout{equation}{labday}


%%%%%%%%%%%%%%%%%%%%%%%%%%%%%%%%%%%%%%%%%%%%%%%
%% EDOC THESIS TEMPLATE: Variant 1.0 -> Latin modern, large text width&height
%%%%%%%%%%%%%%%%%%%%%%%%%%%%%%%%%%%%%%%%%%%%%%%
%\usepackage{lmodern}
%\usepackage[a4paper,top=22mm,bottom=28mm,inner=35mm,outer=25mm]{geometry}
%%%%%%%%%%%%%%%%%%%%%%%%%%%%%%%%%%%%%%%%%%%%%%%

%%%%%%%%%%%%%%%%%%%%%%%%%%%%%%%%%%%%%%%%%%%%%%
% EDOC THESIS TEMPLATE: Variant 2.0 -> Utopia, Gabarrit A (lighter pages)
%%%%%%%%%%%%%%%%%%%%%%%%%%%%%%%%%%%%%%%%%%%%%%
%\usepackage{fourier} % Utopia font-typesetting including mathematical formula compatible with newer TeX-Distributions (>2010)
%\usepackage{utopia} % on older systems -> use this package instead of fourier in combination with mathdesign for better looking results
%\usepackage[adobe-utopia]{mathdesign}

%\setlength{\textwidth}{146.8mm} % = 210mm - 37mm - 26.2mm
%\setlength{\oddsidemargin}{11.6mm} % 37mm - 1in (from hoffset)
%\setlength{\evensidemargin}{0.8mm} % = 26.2mm - 1in (from hoffset)
%\setlength{\topmargin}{-2.2mm} % = 0mm -1in + 23.2mm 
%\setlength{\textheight}{221.9mm} % = 297mm -29.5mm -31.6mm - 14mm (12 to accomodate footline with pagenumber)
%\setlength{\headheight}{14pt}
%%%%%%%%%%%%%%%%%%%%%%%%%%%%%%%%%%%%%%%%%%%%%%

